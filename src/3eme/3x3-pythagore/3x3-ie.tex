\documentclass[11pt]{article}
\usepackage{geometry,marginnote} % Pour passer au format A4
\geometry{hmargin=1cm, vmargin=1.5cm} % 

% Page et encodage
\usepackage[T1]{fontenc} % Use 8-bit encoding that has 256 glyphs
\usepackage[english,french]{babel} % Français et anglais
\usepackage[utf8]{inputenc} 

\usepackage{lmodern}
\usepackage[np]{numprint}
\setlength\parindent{0pt}

% Graphiques
\usepackage{graphicx,float,grffile}
\usepackage{tikz,pst-eucl,pst-plot,pstricks,pst-node,pstricks-add,pst-fun,pgfplots} 

% Maths et divers
\usepackage{amsmath,amsfonts,amssymb,amsthm,verbatim,scratch3}
\usepackage{multicol,enumitem,url,eurosym,gensymb,tabularx}

\DeclareUnicodeCharacter{20AC}{\euro}



% Sections
\usepackage{sectsty} % Allows customizing section commands
\allsectionsfont{\centering \normalfont\scshape}

% Tête et pied de page
\usepackage{fancyhdr} \pagestyle{fancy} \fancyhead{} \fancyfoot{}

%\fancyfoot[L]{Collège Faubert}
%\fancyfoot[C]{\thepage / 6}
%\fancyfoot[R]{Série Générale}

\renewcommand{\headrulewidth}{0pt} % Remove header underlines
%\renewcommand{\footrulewidth}{0pt} % Remove footer underlines

\newcommand{\horrule}[1]{\rule{\linewidth}{#1}} % Create horizontal rule command with 1 argument of height

\newcommand{\Pointilles}[1][3]{%
  \multido{}{#1}{\makebox[\linewidth]{\dotfill}\\[\parskip]
}}

\newtheorem{Definition}{Définition}

\usepackage{siunitx}
\sisetup{
    detect-all,
    output-decimal-marker={,},
    group-minimum-digits = 3,
    group-separator={~},
    number-unit-separator={~},
    inter-unit-product={~}
}

\setlength{\columnseprule}{1pt}


\begin{document}

\begin{center}
  \textit{Pourquoi apprendre alors que l’ignorance est instantanée ?} - \textbf{Bill Watterson}
\end{center}

\begin{multicols}{2} 
\subsection*{Ex1 : Rédaction Pythagore}
\begin{enumerate}
  \item[1.] Le triangle TRC est rectangle en R. On a TR = 23cm et RC = 37cm. Calculer TC.
  \item[2.] Le triangle KLO est rectangle en L. On a KO = 45cm et LO= 29cm. Calculer KL
\end{enumerate} 

\subsection*{Ex2 : Triangle rectangle}
Soit ABC un triangle rectangle en A. On a AB = 28cm et BC = 37cm.
 \begin{enumerate}
  \item[1.] Calculer la longueur AC.
  \item[2.] Calculer le périmètre du triangle ABC.
  \item[3.] Calculer l'aire du triangle ABC. 
\end{enumerate} \columnbreak

\subsection*{Ex3 : Fonctions}
Soit $f$ une fonction de $x$. \\
$f : x \longmapsto x^2 - 10x$

\begin{enumerate}
  \item[1.] Calculer $f(10)$.
  \item[2.] Calculer l'image de $-4$ par la fonction $f$.
\end{enumerate} 

\subsection*{Ex4 : Fonctions}
Soit $g$ une fonction de $x$. \\
$g : x \longmapsto 5x + 7$

\begin{enumerate}
  \item[1.] Calculer $g(10)$.
  \item[2.] Calculer l'image de $-4$ par la fonction $g$.
  \item[3.] Résoudre l'équation pour trouver un antécédent de $712$ par la fonction $g$. 
\end{enumerate}
\end{multicols}

\subsection*{Ex4 : Fonctions}
\textit{Aucune justification n'est demandée. Exercice de lecture de tableau}

 \begin{center}
    \begin{tabular}{|c|c|c|c|c|c|c|c|}
      \hline
      $x$ & $-10$ & $-5$ & $-2$ & $0$ & $2$ & $5$ & $10$ \\  \hline
      $h(x)$ & 5 & 3 & 2 & 4 & -5 & -3 & 0 \\  \hline
    \end{tabular}
  \end{center}

 \begin{enumerate}
  \item[1.] Quelle est l'image de $5$ par la fonction $h$ ?
  \item[2.] Quel nombre a pour image $2$ ?
  \item[3.] Quelle est un antécédent de $0$ par la fonction $h$ ?
  \item[4.] Quel nombre a pour antécédent $4$ ?
\end{enumerate}

\subsection*{Ex5 : Fonctions}
\textit{Aucune justification n'est demandée. Exercice de lecture graphique}

 \begin{figure}[H]
    \centering
    \includegraphics[width=0.3\linewidth]{3x3-pythagore/ex4.pdf}
  \end{figure}

 \begin{enumerate}
  \item[1.] Recopier et compléter le tableau de valeurs. 
 \begin{center}
    \begin{tabular}{|c|c|c|c|c|c|c|c|}
      \hline
      $x$ & $-5$ & $-4$ & $-2$ & $-1$ & $0$ & $2$ & $3$ \\  \hline
      $j(x)$ &  &  &  &  &  &  &  \\  \hline
    \end{tabular}
  \end{center}

  \item[2.] Quelle est l'image de $1$ par la fonction $j$ ?
  \item[3.] Quels sont les antécédents de $2$ par la fonction $j$ ?
  \item[4.] Combien d'antécédents de 0 a la fonction $j$ ? 
\end{enumerate}

\end{document}