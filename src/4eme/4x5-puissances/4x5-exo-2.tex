\documentclass[11pt]{article}
\usepackage{geometry,marginnote} % Pour passer au format A4
\geometry{hmargin=1cm, vmargin=1.5cm} % 

% Page et encodage
\usepackage[T1]{fontenc} % Use 8-bit encoding that has 256 glyphs
\usepackage[english,french]{babel} % Français et anglais
\usepackage[utf8]{inputenc} 

\usepackage{lmodern}
\usepackage[np]{numprint}
\setlength\parindent{0pt}

% Graphiques
\usepackage{graphicx,float,grffile}
\usepackage{tikz,pst-eucl,pst-plot,pstricks,pst-node,pstricks-add,pst-fun,pgfplots} 

% Maths et divers
\usepackage{amsmath,amsfonts,amssymb,amsthm,verbatim,scratch3}
\usepackage{multicol,enumitem,url,eurosym,gensymb,tabularx}

\DeclareUnicodeCharacter{20AC}{\euro}



% Sections
\usepackage{sectsty} % Allows customizing section commands
\allsectionsfont{\centering \normalfont\scshape}

% Tête et pied de page
\usepackage{fancyhdr} \pagestyle{fancy} \fancyhead{} \fancyfoot{}

%\fancyfoot[L]{Collège Faubert}
%\fancyfoot[C]{\thepage / 6}
%\fancyfoot[R]{Série Générale}

\renewcommand{\headrulewidth}{0pt} % Remove header underlines
%\renewcommand{\footrulewidth}{0pt} % Remove footer underlines

\newcommand{\horrule}[1]{\rule{\linewidth}{#1}} % Create horizontal rule command with 1 argument of height

\newcommand{\Pointilles}[1][3]{%
  \multido{}{#1}{\makebox[\linewidth]{\dotfill}\\[\parskip]
}}

\newtheorem{Definition}{Définition}

\usepackage{siunitx}
\sisetup{
    detect-all,
    output-decimal-marker={,},
    group-minimum-digits = 3,
    group-separator={~},
    number-unit-separator={~},
    inter-unit-product={~}
}

\setlength{\columnseprule}{1pt}


\begin{document}

\textbf{Nom, Prénom :} \hspace{8cm} \textbf{Classe :} \hspace{3cm} \textbf{Date :}\\

\textbf{Ex 1 : Écriture scientifique}

\begin{multicols}{2}
  \begin{itemize}[label={$\bullet$}]
  \item $\SI{13400000}{} = \dotfill$
  \item $\SI{6400000000}{} = \dotfill$
  \item $\SI{-1300000000}{} = \dotfill$
  \item $\SI{0,000002}{} = \dotfill$
  \item $\SI{-0,00000000425}{} = \dotfill$
  \item $\SI{0,000006071}{} = \dotfill$
  \end{itemize}
\end{multicols}

\textbf{Ex 2 : Calculer}

\begin{multicols}{2}
  \begin{itemize}[label={$\bullet$}]
  \item $ 2^{12} + 5^2 \times 10^3 = \dotfill$
  \item $ 12^{-5} \times ( 15^3 + 10^{-3}) = \dotfill$
  \end{itemize}
\end{multicols}

\textbf{Ex 3 : Les règles de calculs}

\textit{Écrire le résultat sous la forme d'une puissance en utilisant les règles.}

\begin{multicols}{4}
  \begin{itemize}[label={$\bullet$}]
  \item $6^{7}  \times  6^{8}  =  \dotfill$
  \item $\dfrac{10^{12}}{10^{10}} = \dotfill$
  \item $5^{6} \times 5^{5} = \dotfill$
  \item $\dfrac{11^{11}}{11^{6}} = \dotfill$
  \item $11^{9} \times 7^{9} = \dotfill$
  \item $5^{6} \times 4^{6} = \dotfill$
  \item $(12^{10})^{8} = \dotfill$
  \item $(10^{10})^{7} = \dotfill$
  \end{itemize}
\end{multicols}


\begin{multicols}{2}

\textbf{Ex 4 : Démontrer}

\textit{Écrire la démonstration pour $(2^4)^2 = 2^{8}$} \\
\Pointilles[3]

\textbf{Pb1}

Je possède un sac de 26 millions d’euros en billet de 20 \euro{}. Les billets de banque ont une épaisseur de $60 \times 10^{-6} m$.

\textbf{Quelle hauteur atteindrait la pile de billets ?} \columnbreak

\textbf{Pb2} 

Une molécule de dioxyde de carbone est composée d'un atome de carbone (C) et de deux atomes d'oxygène (O) : $CO_2$. La masse d'un atome de carbone est $ m_c = 2 \times 10^{-26}kg$ et la masse d'un atome d'oxygène est $ m_O = 1,8 \times 10^{-26}kg$. \\

\textbf{Combien trouve-t-on de molécules de dioxyde carbone dans 4 kg ?}

\end{multicols}

\Pointilles[6]

\textbf{Pb3}

\og On place un grain de riz sur la première case d'un échiquier. Si on fait en sorte de doubler à chaque case le nombre de grains de la case précédente : un grain sur la première case, deux sur la deuxième, quatre sur la troisième, etc.,

\begin{itemize}
    \item[1.] Sachant qu'il y a 64 cases, combien de grains de riz obtient-on sur la dernière case ? 
    \item[2.] On estime que le nombre de grain de riz total sur l'échiquier est : $T = 2^{64} - 1$. \\
    Faire ce calcul. La soustraction est-elle effectuée par la calculatrice ?
    \item[3.] La masse d'un grain de riz est $0.05 \times 10^{-3} kg$. Quelle est la masse totale de l'échiquier ? 
\end{itemize}

\Pointilles[6]


\end{document}
