\documentclass[11pt]{article}
\usepackage{geometry,marginnote} % Pour passer au format A4
\geometry{hmargin=1cm, vmargin=1.5cm} % 

% Page et encodage
\usepackage[T1]{fontenc} % Use 8-bit encoding that has 256 glyphs
\usepackage[english,french]{babel} % Français et anglais
\usepackage[utf8]{inputenc} 

\usepackage{lmodern}
\usepackage[np]{numprint}
\setlength\parindent{0pt}

% Graphiques
\usepackage{graphicx,float,grffile}
\usepackage{tikz,pst-eucl,pst-plot,pstricks,pst-node,pstricks-add,pst-fun,pgfplots} 

% Maths et divers
\usepackage{amsmath,amsfonts,amssymb,amsthm,verbatim,scratch3}
\usepackage{multicol,enumitem,url,eurosym,gensymb,tabularx}

\DeclareUnicodeCharacter{20AC}{\euro}



% Sections
\usepackage{sectsty} % Allows customizing section commands
\allsectionsfont{\centering \normalfont\scshape}

% Tête et pied de page
\usepackage{fancyhdr} \pagestyle{fancy} \fancyhead{} \fancyfoot{}

%\fancyfoot[L]{Collège Faubert}
%\fancyfoot[C]{\thepage / 6}
%\fancyfoot[R]{Série Générale}

\renewcommand{\headrulewidth}{0pt} % Remove header underlines
%\renewcommand{\footrulewidth}{0pt} % Remove footer underlines

\newcommand{\horrule}[1]{\rule{\linewidth}{#1}} % Create horizontal rule command with 1 argument of height

\newcommand{\Pointilles}[1][3]{%
  \multido{}{#1}{\makebox[\linewidth]{\dotfill}\\[\parskip]
}}

\newtheorem{Definition}{Définition}

\usepackage{siunitx}
\sisetup{
    detect-all,
    output-decimal-marker={,},
    group-minimum-digits = 3,
    group-separator={~},
    number-unit-separator={~},
    inter-unit-product={~}
}

\setlength{\columnseprule}{1pt}


\begin{document}

\textbf{Nom, Prénom :} \hspace{8cm} \textbf{Classe :} \hspace{3cm} \textbf{Date :}\\


\begin{center}
  \textit{En mathématiques, on ne comprend pas les choses, on s’y habitue..}  - \textbf{John Von Neumann}
\end{center}

\subsection*{Ex 1 : Écriture scientifique}

\begin{multicols}{2}
  \begin{itemize}[label={$\bullet$}]
  \item $\SI{25700000}{} = \dotfill$
  \item $\SI{-26000000000}{} = \dotfill$
  \item $\SI{-0,00000000001335}{} = \dotfill$
  \item $\SI{0,0000009071}{} = \dotfill$
  \end{itemize}
\end{multicols}

\subsection*{Ex 2 : Calculer}

\begin{multicols}{2}
  \begin{itemize}[label={$\bullet$}]
  \item $ 4^{12} + 5^3 \times 10^3 = \dotfill$
  \item $ 9^{-5} \times ( 8^{-3} + 10^{-3}) = \dotfill$
  \end{itemize}
\end{multicols}

\subsection*{Ex 3 : Démontrer que $10^3 \times 10^2 = 10^5$}

\Pointilles[4]

\subsection*{Ex 4 : Les règles de calculs}

\textit{Écrire le résultat sous la forme d'une puissance en utilisant les règles.}

\begin{multicols}{2}
  \begin{itemize}[label={$\bullet$}]
  \item $6^{700}  \times  6^{120}  =  \dotfill$
  \item $\dfrac{10^{436}}{10^{111}} = \dotfill$
  \item $11^{902} \times 7^{902} = \dotfill$
  \item $(10^{100})^{6} = \dotfill$
  \end{itemize}
\end{multicols}

\subsection*{Pb1 - TON618} 

La masse de la terre est $M_T = 1,989 \times 10^{30} \, kg$. \newline
La masse du trou noir TON618 est $\SI{66 000 000 000}{}$ fois plus lourde. \\

Quelle est la masse du trou noir ?

\Pointilles[5]

\subsection*{Pb2 - Célérité}

La lumière se déplace à la vitesse de $3 \times 10^8$ m/s. \\

Quelle distance parcourt-elle en 2 jours ?

\Pointilles[6]

\newpage

\subsection*{Pb3 - Pile d'argent}

Je possède un sac de 520 millions d’euros en billet de 50 \euro{}. Les billets de banque ont une épaisseur de $0,1 \times 10^{-3} \, m$. \\

Quelle hauteur atteindrait la pile de billets ?

\Pointilles[5]

\subsection*{Pb4 - $H_{2}O$} 

Une molécule d'eau est composée d'un atome d'oxygène (O) et de deux atomes d'hydrogène (H) : $H_{2}O$. La masse d'un atome de d'hydrogène est $m_H = 1,67\times 10^{-27} \,kg$ et la masse d'un atome d'oxygène est $ m_O = 2,65 \times 10^{-26} \,kg$. \\

\begin{itemize}
  \item[1.] Quelle est la masse d'une molécule $H_{2}O$ ? 
  \item[2.] Combien trouve-t-on de molécules d'eau dans 2L d'eau ?
\end{itemize}

\Pointilles[7]

\subsection*{Pb5 - mot de passe}

Un pirate informatique met trois plus de temps à déchiffrer un mot de passe par \textit{bruteforce} pour chaque caractère ajouté au mot de passe. Il met 3s pour déchiffre un mot de passe avec un seul caractère.  

\begin{itemize}
  \item[1.] Combien de temps met-il à déchiffrer un mot de passe avec deux caractères ?
  \item[2.] Combien de temps met-il à déchiffrer un mot de passe avec huit caractères ? \\
  Ce temps est-il plus grand que deux heures ?
  \item[3.] Combien de temps met-il à déchiffrer un mot de passe avec 20 caractères ? \\
  Ce temps est-il plus grand que cent ans ?
\end{itemize}


\Pointilles[14]

\end{document}
