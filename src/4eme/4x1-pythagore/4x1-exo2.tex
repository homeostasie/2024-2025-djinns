\documentclass[11pt]{article}
\usepackage{geometry,marginnote} % Pour passer au format A4
\geometry{hmargin=1cm, vmargin=1.5cm} % 

% Page et encodage
\usepackage[T1]{fontenc} % Use 8-bit encoding that has 256 glyphs
\usepackage[english,french]{babel} % Français et anglais
\usepackage[utf8]{inputenc} 

\usepackage{lmodern}
\usepackage[np]{numprint}
\setlength\parindent{0pt}

% Graphiques
\usepackage{graphicx,float,grffile}
\usepackage{tikz,pst-eucl,pst-plot,pstricks,pst-node,pstricks-add,pst-fun,pgfplots} 

% Maths et divers
\usepackage{amsmath,amsfonts,amssymb,amsthm,verbatim,scratch3}
\usepackage{multicol,enumitem,url,eurosym,gensymb,tabularx}

\DeclareUnicodeCharacter{20AC}{\euro}



% Sections
\usepackage{sectsty} % Allows customizing section commands
\allsectionsfont{\centering \normalfont\scshape}

% Tête et pied de page
\usepackage{fancyhdr} \pagestyle{fancy} \fancyhead{} \fancyfoot{}

%\fancyfoot[L]{Collège Faubert}
%\fancyfoot[C]{\thepage / 6}
%\fancyfoot[R]{Série Générale}

\renewcommand{\headrulewidth}{0pt} % Remove header underlines
%\renewcommand{\footrulewidth}{0pt} % Remove footer underlines

\newcommand{\horrule}[1]{\rule{\linewidth}{#1}} % Create horizontal rule command with 1 argument of height

\newcommand{\Pointilles}[1][3]{%
  \multido{}{#1}{\makebox[\linewidth]{\dotfill}\\[\parskip]
}}

\newtheorem{Definition}{Définition}

\usepackage{siunitx}
\sisetup{
    detect-all,
    output-decimal-marker={,},
    group-minimum-digits = 3,
    group-separator={~},
    number-unit-separator={~},
    inter-unit-product={~}
}

\setlength{\columnseprule}{1pt}


\begin{document}
\textbf{Nom, Prénom :} \hspace{8cm} \textbf{Classe :} \hspace{3cm} \textbf{Date :}\\

\textbf{Ex1 - Calculer}

Les triangles sont rectangles. À l'aide du théorème de Pythagore, faire le tableau et trouver les longueurs manquantes. 

\begin{multicols}{3}

\begin{figure}[H]
  \centering
  \includegraphics[width=0.9\linewidth]{4x1-pythagore/ex1b.pdf}
\end{figure} \columnbreak

\Pointilles[8] \columnbreak \\
\Pointilles[8]

\end{multicols}

\textbf{Ex2 - Égalité de Pythagore} - \textit{Écrire les égalités de Pythagore.}

\begin{multicols}{2}
Le triangle VPN est rectangle en N. \\ \Pointilles[1] \\
Le triangle RAM est rectangle en A. \\ \Pointilles[1] 
\end{multicols}

\textbf{Ex3 - Rédaction}

\begin{multicols}{2} \begin{enumerate}
  \item Le triangle LOW est rectangle en L. On a LO = 14cm et LW = 20cm. Calculer OW. \\ \Pointilles[7] \columnbreak
  \item Le triangle UFC est rectangle en F. On a UF = 51cm et CF = 63cm. Calculer UC. \\ \Pointilles[7]
\end{enumerate} \end{multicols} 

\begin{multicols}{2} 

\textbf{Ex4 - Problème} \\

Dans The Wind Waker, Link doit réparer la voile triangulaire de son bateau. Pour cela, il doit coudre un fils le long des trois côtés du triangle. 

\textbf{Calculer le périmètre du triangle formé par la voile.} 

\Pointilles[13] \columnbreak

\begin{figure}[H]
  \centering
  \includegraphics[width=0.8\linewidth]{4x1-pythagore/pb2.pdf}
\end{figure}

\end{multicols}

\end{document}