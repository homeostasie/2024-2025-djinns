\documentclass[11pt]{article}
\usepackage{geometry,marginnote} % Pour passer au format A4
\geometry{hmargin=1cm, vmargin=1.5cm} % 

% Page et encodage
\usepackage[T1]{fontenc} % Use 8-bit encoding that has 256 glyphs
\usepackage[english,french]{babel} % Français et anglais
\usepackage[utf8]{inputenc} 

\usepackage{lmodern}
\usepackage[np]{numprint}
\setlength\parindent{0pt}

% Graphiques
\usepackage{graphicx,float,grffile}
\usepackage{tikz,pst-eucl,pst-plot,pstricks,pst-node,pstricks-add,pst-fun,pgfplots} 

% Maths et divers
\usepackage{amsmath,amsfonts,amssymb,amsthm,verbatim,scratch3}
\usepackage{multicol,enumitem,url,eurosym,gensymb,tabularx}

\DeclareUnicodeCharacter{20AC}{\euro}



% Sections
\usepackage{sectsty} % Allows customizing section commands
\allsectionsfont{\centering \normalfont\scshape}

% Tête et pied de page
\usepackage{fancyhdr} \pagestyle{fancy} \fancyhead{} \fancyfoot{}

%\fancyfoot[L]{Collège Faubert}
%\fancyfoot[C]{\thepage / 6}
%\fancyfoot[R]{Série Générale}

\renewcommand{\headrulewidth}{0pt} % Remove header underlines
%\renewcommand{\footrulewidth}{0pt} % Remove footer underlines

\newcommand{\horrule}[1]{\rule{\linewidth}{#1}} % Create horizontal rule command with 1 argument of height

\newcommand{\Pointilles}[1][3]{%
  \multido{}{#1}{\makebox[\linewidth]{\dotfill}\\[\parskip]
}}

\newtheorem{Definition}{Définition}

\usepackage{siunitx}
\sisetup{
    detect-all,
    output-decimal-marker={,},
    group-minimum-digits = 3,
    group-separator={~},
    number-unit-separator={~},
    inter-unit-product={~}
}

\setlength{\columnseprule}{1pt}


\begin{document}

\begin{center}
  \textit{Pourquoi apprendre alors que l’ignorance est instantanée ?} - \textbf{Bill Watterson}
\end{center}

\subsection*{Ex1  : Théorème de Pythagore}
Réciter le théorème de Pythagore

\subsection*{Ex2 : Calculer}

\begin{multicols}{2} 
\begin{itemize}[label={$\bullet$}]
  \item $25^2 - 15^2$
  \item $(35 - 22)^2$
  \item $\sqrt{13^2 - 3}$
  \item $\sqrt{27} - \sqrt{7}$
\end{itemize}
\end{multicols} 

 \subsection*{Ex3 : Écrire l'égalité}
 \begin{enumerate}
  \item[1.] Le triangle FRT est rectangle en F.
  \item[2.] Le triangle TYU est rectangle en Y.
  \item[3.] Le triangle CVB est rectangle en B. 
\end{enumerate}

 \subsection*{Ex4 : Rédaction}
 \begin{enumerate}
  \item[1.] Le triangle TRC est rectangle en R. On a TR = 23cm et RC = 37cm. Calculer TC.
  \item[2.] Le triangle KLO est rectangle en L. On a KO = 45cm et LO= 29cm. Calculer KL
\end{enumerate}

 \subsection*{PB1 : théorique}
Soit ABC un triangle rectangle en A. On a AB = 28cm et BC = 37cm.
 \begin{enumerate}
  \item[1.] Calculer la longueur AC.
  \item[2.] Calculer le périmètre du triangle ABC.
  \item[3.] Calculer l'aire du triangle ABC. 
\end{enumerate}

 \subsection*{PB2 : Mario}

 \begin{multicols}{2} 
 \begin{figure}[H]
    \centering
    \includegraphics[width=0.3\linewidth]{4x1-pythagore/pb4-mario.png}
  \end{figure}

  \begin{figure}[H]
    \centering
    \includegraphics[width=0.6\linewidth]{4x1-pythagore/pb4.pdf}
  \end{figure}
\end{multicols}

Mario souhaite sauver la princesse Peach. Elle est prisonnière en haut d'une tour. Son plan est de lancer un grappin par dessus les douves et que celui-ci s’agrippe directement en haut de la tour. 

\begin{itemize}[label={$\bullet$}]
  \item La tour : DT = 45m de haut
  \item Les douves : GD = 35m de long.
\end{itemize}

Quelle longueur doit-il prévoir pour son grappin ?

\end{document}