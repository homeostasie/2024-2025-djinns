\documentclass[11pt]{article}
\usepackage{geometry,marginnote} % Pour passer au format A4
\geometry{hmargin=1cm, vmargin=1.5cm} % 

% Page et encodage
\usepackage[T1]{fontenc} % Use 8-bit encoding that has 256 glyphs
\usepackage[english,french]{babel} % Français et anglais
\usepackage[utf8]{inputenc} 

\usepackage{lmodern}
\usepackage[np]{numprint}
\setlength\parindent{0pt}

% Graphiques
\usepackage{graphicx,float,grffile}
\usepackage{tikz,pst-eucl,pst-plot,pstricks,pst-node,pstricks-add,pst-fun,pgfplots} 

% Maths et divers
\usepackage{amsmath,amsfonts,amssymb,amsthm,verbatim,scratch3}
\usepackage{multicol,enumitem,url,eurosym,gensymb,tabularx}

\DeclareUnicodeCharacter{20AC}{\euro}



% Sections
\usepackage{sectsty} % Allows customizing section commands
\allsectionsfont{\centering \normalfont\scshape}

% Tête et pied de page
\usepackage{fancyhdr} \pagestyle{fancy} \fancyhead{} \fancyfoot{}

%\fancyfoot[L]{Collège Faubert}
%\fancyfoot[C]{\thepage / 6}
%\fancyfoot[R]{Série Générale}

\renewcommand{\headrulewidth}{0pt} % Remove header underlines
%\renewcommand{\footrulewidth}{0pt} % Remove footer underlines

\newcommand{\horrule}[1]{\rule{\linewidth}{#1}} % Create horizontal rule command with 1 argument of height

\newcommand{\Pointilles}[1][3]{%
  \multido{}{#1}{\makebox[\linewidth]{\dotfill}\\[\parskip]
}}

\newtheorem{Definition}{Définition}

\usepackage{siunitx}
\sisetup{
    detect-all,
    output-decimal-marker={,},
    group-minimum-digits = 3,
    group-separator={~},
    number-unit-separator={~},
    inter-unit-product={~}
}

\setlength{\columnseprule}{1pt}


\begin{document}

\begin{center}
  \textit{Un tableau ne vit que par celui qui le regarde.} - \textbf{Pablo Picasso}
\end{center}

{\Large \textsc{Éval bilan Semestre 1 - Mathématiques}}

\begin{itemize}
    \item L'usage de la calculatrice de type collège est autorisé. L'usage de tout autre document est interdit. 
    \item Les réponses des exercices 2 et 3 doivent êtres justifiées.
  \end{itemize}

\horrule{2px}

\subsection*{Ex1 - QCM}

\textit{Il faut répondre sur la copie. Une seule réponse juste. Aucune justification demandée pour le QCM.}

\begin{center} \renewcommand{\arraystretch}{2.5} \begin{tabular}{|p{3cm}|p{2cm}|p{2cm}|p{2cm}|p{2cm}|}  \hline
      & A & B & C & D \\ \hline
  $(10 + 2 \times -3)^2$  & 16 & -8 & 1296 & -108 \\ \hline
  $-2 -2 \times -2$       & -6 &  2 &   -8 & 0  \\ \hline
  $\dfrac{2}{5} + \dfrac{3}{5} \times \dfrac{7}{2}$ & $\dfrac{23}{15}$ & $\dfrac{35}{10}$ & $\dfrac{25}{10}$ & $\dfrac{1}{2} $ \\ \hline
  $\dfrac{3}{8} \div \dfrac{5}{12}$ & $ \dfrac{19}{10} $ & $ \dfrac{3}{8} \times \dfrac{12}{5} $ & 9 & $ \dfrac{19}{24} $  \\ \hline
\end{tabular} \end{center}

\begin{multicols}{2}\noindent
\subsection*{Ex2 - Géométrie}

\begin{figure}[H]
  \centering
  \includegraphics[width=0.8\linewidth]{4xDM/ex2.pdf}
\end{figure}

On a $AC = 15cm$.

\begin{enumerate}
  \item[1.] Calculer AB.
  \item[2.] Calculer BC.
  \item[3.] Calculer le périmètre de la figure.
  \item[4.] Calculer l'aire de la figure. 
\end{enumerate} \columnbreak


\subsection*{Ex3 - Énergie}

\begin{enumerate}
  \item[1.] La production d’énergie primaire en la France s’établit à $2 \, 950$ TWh en 2023. \\
  La répartition est la suivante : 

  \begin{itemize}[label={$\bullet$}]
    \item Nucléaire : 43\%. 
    \item Pétrole : 28\%. 
    \item Énergie Renouvelable : 15\%. 
    \item Nucléaire : 14\%.
  \end{itemize} 
  Calculer la production d'énergie pour chaque source. 

  \item[2.] La production d'énergie renouvelable est 442,5 TWh. \\
  Elle est répartie à partir de différentes sources.

  \begin{itemize}[label={$\bullet$}]
    \item Hydraulique : 150 TWh. 
    \item Solaire : 180 TWh. 
    \item Éolien : 65 TWh.
    \item Autre : le reste. 
  \end{itemize} 

  Calculer le pourcentage de chaque source d'énergie renouvelable par rapport à la production d'énergie renouvelable.

  \item[3.] Pour 2025, La France prévoit d'augmenter sa production d'énergie renouvelable de 15\%. Calculer cette prévision. 
\end{enumerate}

\end{multicols}


\end{document}