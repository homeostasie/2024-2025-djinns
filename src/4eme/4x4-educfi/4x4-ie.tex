\documentclass[11pt]{article}
\usepackage{geometry,marginnote} % Pour passer au format A4
\geometry{hmargin=1cm, vmargin=1.5cm} % 

% Page et encodage
\usepackage[T1]{fontenc} % Use 8-bit encoding that has 256 glyphs
\usepackage[english,french]{babel} % Français et anglais
\usepackage[utf8]{inputenc} 

\usepackage{lmodern}
\usepackage[np]{numprint}
\setlength\parindent{0pt}

% Graphiques
\usepackage{graphicx,float,grffile}
\usepackage{tikz,pst-eucl,pst-plot,pstricks,pst-node,pstricks-add,pst-fun,pgfplots} 

% Maths et divers
\usepackage{amsmath,amsfonts,amssymb,amsthm,verbatim,scratch3}
\usepackage{multicol,enumitem,url,eurosym,gensymb,tabularx}

\DeclareUnicodeCharacter{20AC}{\euro}



% Sections
\usepackage{sectsty} % Allows customizing section commands
\allsectionsfont{\centering \normalfont\scshape}

% Tête et pied de page
\usepackage{fancyhdr} \pagestyle{fancy} \fancyhead{} \fancyfoot{}

%\fancyfoot[L]{Collège Faubert}
%\fancyfoot[C]{\thepage / 6}
%\fancyfoot[R]{Série Générale}

\renewcommand{\headrulewidth}{0pt} % Remove header underlines
%\renewcommand{\footrulewidth}{0pt} % Remove footer underlines

\newcommand{\horrule}[1]{\rule{\linewidth}{#1}} % Create horizontal rule command with 1 argument of height

\newcommand{\Pointilles}[1][3]{%
  \multido{}{#1}{\makebox[\linewidth]{\dotfill}\\[\parskip]
}}

\newtheorem{Definition}{Définition}

\usepackage{siunitx}
\sisetup{
    detect-all,
    output-decimal-marker={,},
    group-minimum-digits = 3,
    group-separator={~},
    number-unit-separator={~},
    inter-unit-product={~}
}

\setlength{\columnseprule}{1pt}


\begin{document}

\begin{center}
  \textit{La réalité, c'est ce qui refuse de disparaître quand on cesse d'y croire.}  - \textbf{Philip K. Dick}
\end{center}

\begin{multicols}{3}\noindent
  \textbf{Ex1 - Appliquer un pourcentage}\\

  \begin{itemize}[label={$\bullet$}]
    \item 40\% de 3\,500€
    \item 32\% de 4\,100€
    \item 6\% de 12\,000€
    \item 92\% de 24\,000€
  \end{itemize}

  \textbf{Ex2 - Calculer une pourcentage}\\

  \begin{itemize}[label={$\bullet$}]
    \item 530€ par rapport à 2\,100€
    \item 2\,530€ par rapport à 43\,000€
    \item 530€ par rapport à 700€
    \item 530€ par rapport à 7\,500€
  \end{itemize}

\textbf{Ex3 - Calculer une augmentation ou une réduction}\\


\begin{itemize}[label={$\bullet$}]
  \item  8\% d'augmentation sur 2\,200€
  \item 12\% d'augmentation sur 3\,500€
  \item 20\% de réduction sur 4\,800€
  \item 40\% d'augmentation 1\,800€
\end{itemize}
\end{multicols}

\textbf{Pb1 - construction}\\

Je souhaite vendre mon terrain de $5\,600m^2$. Pour en tirer le plus d'argent de possible. Je souhaite faire construire un petit immeuble dessus et vendre individuellement les appartements. Je réalise une étude de marché en amont. 

\begin{enumerate}
  \item[1.] J'ai le choix entre trois propositions : 

    \begin{itemize}[label={$\bullet$}]
        \item 25 appartements à 160\,000€
        \item 18 appartements à 250\,000€
        \item 4 maisons individuelles à 1\,000\,000€
    \end{itemize}
    Quelle proposition est la plus avantageuse ?
  
  \item[2.] Je choisis la proposition 2. Je pense pouvoir récupérer 4\,500\,000€ à la fin du projet. \newline
    Je fait le choix d'un entrepreneur pour réaliser ce projet. Son devis est 3 145 000€ pour 2 ans de travaux. \\
  Quel bénéfice vais-je pouvoir tirer de la vente des appartements ?

  \item[3.] Je me place du point de vue d'entreprise qui va construire l'immeuble. \\
    Quelle est la recette de l'entreprise sur ce projet ?

  \item[4.] Pour construire l'immeuble demandé, j'ai besoin :
    \begin{itemize}[label={$\bullet$}]
      \item 8 employés : 41\,000€ par an, par employé
      \item Location d'engins : 180\,000 pour deux ans.
      \item Matériaux : 850\,000€ au total.
      \item Assurance : 60\,000€ par an.
    \end{itemize}
  Quelle est la dépense totale pour ce projet ?

  \item[5.] Quel est le bénéfice généré par l'entreprise sur ce projet de deux ans ? \\

\end{enumerate}

\textbf{Pb2 - les salaires}\\

Dans ma PME de carrosserie, nous sommes deux : moi et un employé. À la fin de chaque mois, je partage les recettes. Sur le mois d'Octobre, l'entreprise récupère 12\,500€ de recettes. \newline
Le partage est le suivant : 
\begin{itemize}[label={$\bullet$}]
  \item 36\% pour mon employé
  \item 40\% pour mon salaire
  \item Le reste est réservé pour la gestion de l'entreprise
\end{itemize}

\begin{enumerate}
  \item[1.] Calculer chaque montant.
  \item[2.] Mon salaire et celui de mon employé sont indiqués en brut. Pour avoir le net, je dois payés des cotisations.
  \begin{itemize}[label={$\bullet$}]
    \item Cotisation patronale : 12\%.
    \item Cotisation salariale : 10\%.
  \end{itemize}
  Calculer les salaires nets de chacun ?
  \item[3.] Sur l'année, le chiffre d'affaire de mon entreprise est 850\,000€. Je dois reverser 5\% du chiffre d'affaire à l'URSSAF. \\
  Calculer ce montant.
  \item[4.] Je suis très content de mon employé. Son salaire annuel perçu est 51\,000€. Je souhaite lui offrir une prime de 1\,500€. \\
  Calculer le pourcentage correspondant. 
\end{enumerate}

\newpage

\begin{center}
  \textit{La réalité, c'est ce qui refuse de disparaître quand on cesse d'y croire.}  - \textbf{Philip K. Dick}
\end{center}

\begin{multicols}{3}\noindent
  \textbf{Ex1 - Appliquer un pourcentage}\\

  \begin{itemize}[label={$\bullet$}]
    \item 60\% de 4\,500€
    \item 42\% de 5\,100€
    \item 8\% de 13\,000€
    \item 96\% de 54\,000€
  \end{itemize}

  \textbf{Ex2 - Calculer une pourcentage}\\

  \begin{itemize}[label={$\bullet$}]
    \item 860€ par rapport à 1\,900€
    \item 3\,630€ par rapport à 36\,000€
    \item 830€ par rapport à 900€
    \item 830€ par rapport à 9\,500€
  \end{itemize}

\textbf{Ex3 - Calculer une augmentation ou une réduction}\\


\begin{itemize}[label={$\bullet$}]
  \item  6\% d'augmentation sur 3\,200€
  \item 14\% d'augmentation sur 2\,800€
  \item 30\% de réduction sur 5\,200€
  \item 40\% d'augmentation 2\,200€
\end{itemize}
\end{multicols}

\textbf{Pb1 - construction}\\

Je souhaite vendre mon terrain de $4\,800m^2$. Pour en tirer le plus d'argent de possible. Je souhaite faire construire un petit immeuble dessus et vendre individuellement les appartements. Je réalise une étude de marché en amont. 

\begin{enumerate}
  \item[1.] J'ai le choix entre trois propositions : 

    \begin{itemize}[label={$\bullet$}]
        \item 24 appartements à 180\,000€
        \item 16 appartements à 280\,000€
        \item 4 maisons individuelles à 1\,000\,000€
    \end{itemize}
    Quelle proposition est la plus avantageuse ?
  
  \item[2.] Je choisis la proposition 2. Je pense pouvoir récupérer 4\,480\,000€ à la fin du projet. \newline
    Je fait le choix d'un entrepreneur pour réaliser ce projet. Son devis est 3 264 000€ pour 2 ans de travaux. \\
  Quel bénéfice vais-je pouvoir tirer de la vente des appartements ?

  \item[3.] Je me place du point de vue d'entreprise qui va construire l'immeuble. \\
    Quelle est la recette de l'entreprise sur ce projet ?

  \item[4.] Pour construire l'immeuble demandé, j'ai besoin :
    \begin{itemize}[label={$\bullet$}]
      \item 9 employés : 39\,000€ par an, par employé
      \item Location d'engins : 160\,000 pour deux ans.
      \item Matériaux : 950\,000€ au total.
      \item Assurance : 80\,000€ par an.
    \end{itemize}
  Quelle est la dépense totale pour ce projet ?

  \item[5.] Quel est le bénéfice généré par l'entreprise sur ce projet de deux ans ? \\

\end{enumerate}

\textbf{Pb2 - les salaires}\\

Dans ma PME de carrosserie, nous sommes deux : moi et un employé. À la fin de chaque mois, je partage les recettes. Sur le mois d'Octobre, l'entreprise récupère 14\,200€ de recettes. \newline
Le partage est le suivant : 
\begin{itemize}[label={$\bullet$}]
  \item 36\% pour mon employé
  \item 42\% pour mon salaire
  \item Le reste est réservé pour la gestion de l'entreprise
\end{itemize}

\begin{enumerate}
  \item[1.] Calculer chaque montant.
  \item[2.] Mon salaire et celui de mon employé sont indiqués en brut. Pour avoir le net, je dois payés des cotisations.
  \begin{itemize}[label={$\bullet$}]
    \item Cotisation patronale : 14\%.
    \item Cotisation salariale : 12\%.
  \end{itemize}
  Calculer les salaires nets de chacun ?
  \item[3.] Sur l'année, le chiffre d'affaire de mon entreprise est 780\,000€. Je dois reverser 5\% du chiffre d'affaire à l'URSSAF. \\
  Calculer ce montant.
  \item[4.] Je suis très content de mon employé. Son salaire annuel perçu est 54\,000€. Je souhaite lui offrir une prime de 2\,200€. \\
  Calculer le pourcentage correspondant. 
\end{enumerate}

\end{document}