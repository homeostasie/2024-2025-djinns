\documentclass[11pt]{article}
\usepackage{geometry,marginnote} % Pour passer au format A4
\geometry{hmargin=1cm, vmargin=1.5cm} % 

% Page et encodage
\usepackage[T1]{fontenc} % Use 8-bit encoding that has 256 glyphs
\usepackage[english,french]{babel} % Français et anglais
\usepackage[utf8]{inputenc} 

\usepackage{lmodern}
\usepackage[np]{numprint}
\setlength\parindent{0pt}

% Graphiques
\usepackage{graphicx,float,grffile}
\usepackage{tikz,pst-eucl,pst-plot,pstricks,pst-node,pstricks-add,pst-fun,pgfplots} 

% Maths et divers
\usepackage{amsmath,amsfonts,amssymb,amsthm,verbatim,scratch3}
\usepackage{multicol,enumitem,url,eurosym,gensymb,tabularx}

\DeclareUnicodeCharacter{20AC}{\euro}



% Sections
\usepackage{sectsty} % Allows customizing section commands
\allsectionsfont{\centering \normalfont\scshape}

% Tête et pied de page
\usepackage{fancyhdr} \pagestyle{fancy} \fancyhead{} \fancyfoot{}

%\fancyfoot[L]{Collège Faubert}
%\fancyfoot[C]{\thepage / 6}
%\fancyfoot[R]{Série Générale}

\renewcommand{\headrulewidth}{0pt} % Remove header underlines
%\renewcommand{\footrulewidth}{0pt} % Remove footer underlines

\newcommand{\horrule}[1]{\rule{\linewidth}{#1}} % Create horizontal rule command with 1 argument of height

\newcommand{\Pointilles}[1][3]{%
  \multido{}{#1}{\makebox[\linewidth]{\dotfill}\\[\parskip]
}}

\newtheorem{Definition}{Définition}

\usepackage{siunitx}
\sisetup{
    detect-all,
    output-decimal-marker={,},
    group-minimum-digits = 3,
    group-separator={~},
    number-unit-separator={~},
    inter-unit-product={~}
}

\setlength{\columnseprule}{1pt}


\begin{document}

\begin{center}
  \textit{La vie c’est comme une bicyclette, il faut avancer pour ne pas perdre l’équilibre.} - \textbf{Albert Einstein}
\end{center}


\subsubsection*{Définition}

Un tableau est proportionnel si ...

\subsubsection*{Ex1 - Calculer} 

\textit{Les tableaux sont proportionnels.} Calculer les nombres manquants. (Écrire les calculs.)

\begin{multicols}{2}\noindent
  \begin{center} \begin{tabular}{|c|c|c|}  \hline
    6  &                                55 & $\phantom{\dfrac{azertyuiop}{O}}$ \\  \hline
    14 & $\phantom{\dfrac{azertyuiop}{O}}$ & 148\\  \hline
    \end{tabular} \end{center}

  \begin{center} \begin{tabular}{|c|c|c|}  \hline
    $\phantom{\dfrac{azertyuiop}{O}}$ &  75 & 84 \\  \hline
                                   24 & 106 & $\phantom{\dfrac{azertyuiop}{O}}$\\  \hline
  \end{tabular} \end{center}

\end{multicols}

\subsubsection*{Ex2 : Démontrer} 

\textit{Les tableaux sont-ils proportionnels ?} (Écrire les calculs.) 

\begin{multicols}{2}\noindent
  \begin{center}\begin{tabular}{|c|c|c|} \hline
    24 & 96 & 312 \\  \hline
    2 & 8 & 26\\  \hline
  \end{tabular}\end{center}
\columnbreak 
  \begin{center}\begin{tabular}{|c|c|c|} \hline
    12 & 180 & 61 \\  \hline
    30 & 450 & 150\\  \hline
  \end{tabular}\end{center}

\end{multicols}

\subsubsection*{Pb1 - Vélo custom}

Je souhaite améliorer mon vélo en changeant la selle et en installant deux disques de frein et les quatre plaquettes de frein correspondantes. \\

Je me rends à Décathlon. Le magasin me propose de réaliser toutes les améliorations pour le prix de 430€. Je souhaite faire des économies. J'achète tout le nécessaire et bricole les opérations moi-même. \\

Les prix à l'unité sont : 48€ pour la selle, 70€ pour un disque, 25€ pour une plaquette de frein.\\

Quelles économies vais-je faire ?

\subsubsection*{Pb2 - tapisserie}

Je souhaite améliorer la décoration de chez moi en posant de la tapisserie. Un magasin de papier peints et de tapisseries vend une très beau rouleau de tapisserie au mètre : 3 mètres coûtent 48€.

\begin{enumerate}
  \item[a.] Calculer le prix de 26 mètres de tissu.
  \item[b.] Calculer la longueur du rouleau qu'on peut acheter avec 350€.
\end{enumerate}  


\subsubsection*{Pb3 - peinture} 

Je souhaite repeindre l'extérieur de ma maison en orange. Pour obtenir un pot de peinture orange de 10 litres, je dois mélanger 6 litres de peinture jaune et 4 litres de peinture rouge.\\ 

Avec un pot de 10 litres, je peux peindre une surface de $24m^2$. \\

Un litre de peinture jaune coûte 6€ et un litre de peinture rouge coûte 5€. 

\begin{enumerate}
  \item[a.] Combien de pot orange de 10 litres doit-on avoir afin de recouvrir une surface totale de $250m^2$ ?
  \item[b.] Calculer la quantité de peinture rouge et de peinture jaune nécessaire.
  \item[c.] Calculer le prix total pour réaliser cette opération.  
\end{enumerate}

\newpage


\begin{center}
  \textit{La vie c’est comme une bicyclette, il faut avancer pour ne pas perdre l’équilibre.} - \textbf{Albert Einstein}
\end{center}


\subsubsection*{Définition}

Un tableau est proportionnel si ...

\subsubsection*{Ex1 - Calculer} 

\textit{Les tableaux sont proportionnels.} Calculer les nombres manquants. (Écrire les calculs.) 

\begin{multicols}{2}\noindent
  \begin{center} \begin{tabular}{|c|c|c|}  \hline
    8  &                                75 & $\phantom{\dfrac{azertyuiop}{O}}$ \\  \hline
    13 & $\phantom{\dfrac{azertyuiop}{O}}$ & 232\\  \hline
    \end{tabular} \end{center}

  \begin{center} \begin{tabular}{|c|c|c|}  \hline
    $\phantom{\dfrac{azertyuiop}{O}}$ & 15 & 54 \\  \hline
                                  198 & 76 & $\phantom{\dfrac{azertyuiop}{O}}$\\  \hline
  \end{tabular} \end{center}

\end{multicols}

\subsubsection*{Ex2 : Démontrer} 

\textit{Les tableaux sont-ils proportionnels ?} (Écrire les calculs.) 

\begin{multicols}{2}\noindent
  \begin{center}\begin{tabular}{|c|c|c|} \hline
    26 & 105 & 338 \\  \hline
    2 & 8 & 26\\  \hline
  \end{tabular}\end{center}
\columnbreak 
  \begin{center}\begin{tabular}{|c|c|c|} \hline
    17 & 204 & 153 \\  \hline
    30 & 362 & 270\\  \hline
  \end{tabular}\end{center}

\end{multicols}

\subsubsection*{Pb1 - Vélo custom}

Je souhaite améliorer mon vélo en changeant la selle et en installant deux disques de frein et les quatre plaquettes de frein correspondantes. \\

Je me rends à Décathlon. Le magasin me propose de réaliser toutes les améliorations pour le prix de 410€. Je souhaite faire des économies. J'achète tout le nécessaire et bricole les opérations moi-même. \\

Les prix à l'unité sont : 56€ pour la selle, 65€ pour un disque, 20€ pour une plaquette de frein.\\

Quelles économies vais-je faire ?

\subsubsection*{Pb2 - tapisserie}

Je souhaite améliorer la décoration de chez moi en posant de la tapisserie. Un magasin de papier peints et de tapisseries vend une très beau rouleau de tapisserie au mètre : 4 mètres coûtent 45€.

\begin{enumerate}
  \item[a.] Calculer le prix de 50 mètres de tissu.
  \item[b.] Calculer la longueur du rouleau qu'on peut acheter avec 650€.
\end{enumerate}  


\subsubsection*{Pb3 - peinture} 

Je souhaite repeindre l'extérieur de ma maison en orange. Pour obtenir un pot de peinture orange de 10 litres, je dois mélanger 7 litres de peinture jaune et 3 litres de peinture rouge.\\ 

Avec un pot de 10 litres, je peux peindre une surface de $26m^2$. \\

Un litre de peinture jaune coûte 5€ et un litre de peinture rouge coûte 6€. 

\begin{enumerate}
  \item[a.] Combien de pot orange de 10 litres doit-on avoir afin de recouvrir une surface totale de $310m^2$ ?
  \item[b.] Calculer la quantité de peinture rouge et de peinture jaune nécessaire.
  \item[c.] Calculer le prix total pour réaliser cette opération.  
\end{enumerate}

\end{document}