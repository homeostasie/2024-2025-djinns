\documentclass[11pt]{article}
\usepackage{geometry,marginnote} % Pour passer au format A4
\geometry{hmargin=1cm, vmargin=1.5cm} % 

% Page et encodage
\usepackage[T1]{fontenc} % Use 8-bit encoding that has 256 glyphs
\usepackage[english,french]{babel} % Français et anglais
\usepackage[utf8]{inputenc} 

\usepackage{lmodern}
\usepackage[np]{numprint}
\setlength\parindent{0pt}

% Graphiques
\usepackage{graphicx,float,grffile}
\usepackage{tikz,pst-eucl,pst-plot,pstricks,pst-node,pstricks-add,pst-fun,pgfplots} 

% Maths et divers
\usepackage{amsmath,amsfonts,amssymb,amsthm,verbatim,scratch3}
\usepackage{multicol,enumitem,url,eurosym,gensymb,tabularx}

\DeclareUnicodeCharacter{20AC}{\euro}



% Sections
\usepackage{sectsty} % Allows customizing section commands
\allsectionsfont{\centering \normalfont\scshape}

% Tête et pied de page
\usepackage{fancyhdr} \pagestyle{fancy} \fancyhead{} \fancyfoot{}

%\fancyfoot[L]{Collège Faubert}
%\fancyfoot[C]{\thepage / 6}
%\fancyfoot[R]{Série Générale}

\renewcommand{\headrulewidth}{0pt} % Remove header underlines
%\renewcommand{\footrulewidth}{0pt} % Remove footer underlines

\newcommand{\horrule}[1]{\rule{\linewidth}{#1}} % Create horizontal rule command with 1 argument of height

\newcommand{\Pointilles}[1][3]{%
  \multido{}{#1}{\makebox[\linewidth]{\dotfill}\\[\parskip]
}}

\newtheorem{Definition}{Définition}

\usepackage{siunitx}
\sisetup{
    detect-all,
    output-decimal-marker={,},
    group-minimum-digits = 3,
    group-separator={~},
    number-unit-separator={~},
    inter-unit-product={~}
}

\setlength{\columnseprule}{1pt}


\begin{document}

\begin{center}
  \textit{Le plus grand ennemi de la connaissance n'est pas l'ignorance. C'est l'illusion de la connaissance.} 
  
  \textbf{Stephen Hawking}
\end{center}

\subsection*{Ex : Calculer} 


\begin{itemize}[label={$\bullet$}]
  \item $24 + 6 \times 5$ \\
  \item $5 \times (24 - 20)$ \\
  \item $24 - 12 - 10$ \\ 
  \item $(14 - 6) \times (5 + 35)$ \\
  \item $100 + 2 \times (5 + 10)$ \\ 
  \item $11 \times 6 - 7 \times (13 - 10)$ 
\end{itemize}


\subsection*{pb1 : Inscription} 

25 élèves sont inscrits au club de foot : FC Faubert. Le montant de l'inscription est de 230€ par an par personne.\\ 

Quelle est la somme d'argent perçue par le club ?


\subsection*{pb2 : Équipement} 

En début d'année, l'équipe de vie scolaire passe une commande à Décathlon pour l'achat d'équipement sportif.

\begin{itemize}[label={$\bullet$}]
  \item 5 ballons de foot : 15€ l'unité.
  \item 4 ballons de basket : 22€ l'unité.
  \item 2 cages : 75 l'ensemble. 
  \item 20 maillots : 60€ le lot. 
\end{itemize} 

Calculer le montant de la commande.

\subsection*{pb3 : Boxe} 

Le club de boxe : Boxing Squad Faubert à fait le plein d'adhérents pour cette année et a récupéré la somme d'argent de 6200€ grâce aux inscriptions. Mais il va avoir des dépenses. 

\begin{itemize}[label={$\bullet$}]
  \item 1500€ par an pour la location de la salle de sport.
  \item 1100€ pour un nouveau ring.
  \item 900€ pour divers équipements : gants, sacs, poids, ...
  \item 40€ par mois pendant un an pour l'assurance. 
\end{itemize}

Quelle est la somme d'argent disponible pour le club en tenant compte des dépenses ?

\end{document}