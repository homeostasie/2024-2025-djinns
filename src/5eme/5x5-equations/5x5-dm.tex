\documentclass[11pt]{article}
\usepackage{geometry,marginnote} % Pour passer au format A4
\geometry{hmargin=1cm, vmargin=1.5cm} % 

% Page et encodage
\usepackage[T1]{fontenc} % Use 8-bit encoding that has 256 glyphs
\usepackage[english,french]{babel} % Français et anglais
\usepackage[utf8]{inputenc} 

\usepackage{lmodern}
\usepackage[np]{numprint}
\setlength\parindent{0pt}

% Graphiques
\usepackage{graphicx,float,grffile}
\usepackage{tikz,pst-eucl,pst-plot,pstricks,pst-node,pstricks-add,pst-fun,pgfplots} 

% Maths et divers
\usepackage{amsmath,amsfonts,amssymb,amsthm,verbatim,scratch3}
\usepackage{multicol,enumitem,url,eurosym,gensymb,tabularx}

\DeclareUnicodeCharacter{20AC}{\euro}



% Sections
\usepackage{sectsty} % Allows customizing section commands
\allsectionsfont{\centering \normalfont\scshape}

% Tête et pied de page
\usepackage{fancyhdr} \pagestyle{fancy} \fancyhead{} \fancyfoot{}

%\fancyfoot[L]{Collège Faubert}
%\fancyfoot[C]{\thepage / 6}
%\fancyfoot[R]{Série Générale}

\renewcommand{\headrulewidth}{0pt} % Remove header underlines
%\renewcommand{\footrulewidth}{0pt} % Remove footer underlines

\newcommand{\horrule}[1]{\rule{\linewidth}{#1}} % Create horizontal rule command with 1 argument of height

\newcommand{\Pointilles}[1][3]{%
  \multido{}{#1}{\makebox[\linewidth]{\dotfill}\\[\parskip]
}}

\newtheorem{Definition}{Définition}

\usepackage{siunitx}
\sisetup{
    detect-all,
    output-decimal-marker={,},
    group-minimum-digits = 3,
    group-separator={~},
    number-unit-separator={~},
    inter-unit-product={~}
}

\setlength{\columnseprule}{1pt}


\begin{document}


\horrule{2px}
\section*{DM - Modélisation du monde}
\horrule{2px}

Vous devez choisir une équation parmi celles proposées ci-dessous.

\begin{itemize}[label={$\bullet$}]
  \item Écrire le nom de l'équation comme titre de la feuille.
  \item Écrire l'équation correspondante.
  \item Faire une recherche d'images : coller des images correspondantes à l'équation / faire des desssins pour illustrer.
  \item Faire une recherche sur wikipédia et ré-écrire en quelques lignes ce que représente l'équation. 
\end{itemize} 


\textit{Les équations avec (*) sont un peu plus complexes à illustrer.}

\subsection*{Équation de la chaleur}

$$ \frac {\partial \phi }{\partial t} +\nabla \cdot (\phi \mathbf {V} )=S $$

\subsection*{Équation des ondes}

$$ \nabla ^{2}{\vec {E}} = {\frac {1}{c^{2}}}{\frac {\partial ^{2}{\vec {E}}}{\partial t^{2}}} $$

\subsection*{Loi des gaz parfaits (*)}

$$ PV = nRT $$


\subsection*{Équation d'Einstein (*)}

$$ E = mc^2 $$

\subsection*{Équation de la chute libre}

$$  \frac {\mathrm {d} ^{2}x(t)}{\mathrm {d} t^{2}} = -g $$

\subsection*{Désintégration radioactive (*)}

$$ N(t)=N_{0}\,e^{-{\lambda }t}$$


\subsection*{Équation de Navier Stokes (*)}

$$ \rho \left({\dfrac {\partial e}{\partial t}}+\mathbf {V} \cdot \mathbf {\nabla } e\right)={\mathsf {P}}:\mathbf {\nabla } \mathbf {V} +\mathbf {\nabla } \cdot \mathbf {q} +\mathbf {\nabla } \cdot \mathbf {q} _{R}$$



\end{document}