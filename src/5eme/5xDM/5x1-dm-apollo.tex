\documentclass[11pt]{article}
\usepackage{geometry,marginnote} % Pour passer au format A4
\geometry{hmargin=1cm, vmargin=1.5cm} % 

% Page et encodage
\usepackage[T1]{fontenc} % Use 8-bit encoding that has 256 glyphs
\usepackage[english,french]{babel} % Français et anglais
\usepackage[utf8]{inputenc} 

\usepackage{lmodern}
\usepackage[np]{numprint}
\setlength\parindent{0pt}

% Graphiques
\usepackage{graphicx,float,grffile}
\usepackage{tikz,pst-eucl,pst-plot,pstricks,pst-node,pstricks-add,pst-fun,pgfplots} 

% Maths et divers
\usepackage{amsmath,amsfonts,amssymb,amsthm,verbatim,scratch3}
\usepackage{multicol,enumitem,url,eurosym,gensymb,tabularx}

\DeclareUnicodeCharacter{20AC}{\euro}



% Sections
\usepackage{sectsty} % Allows customizing section commands
\allsectionsfont{\centering \normalfont\scshape}

% Tête et pied de page
\usepackage{fancyhdr} \pagestyle{fancy} \fancyhead{} \fancyfoot{}

%\fancyfoot[L]{Collège Faubert}
%\fancyfoot[C]{\thepage / 6}
%\fancyfoot[R]{Série Générale}

\renewcommand{\headrulewidth}{0pt} % Remove header underlines
%\renewcommand{\footrulewidth}{0pt} % Remove footer underlines

\newcommand{\horrule}[1]{\rule{\linewidth}{#1}} % Create horizontal rule command with 1 argument of height

\newcommand{\Pointilles}[1][3]{%
  \multido{}{#1}{\makebox[\linewidth]{\dotfill}\\[\parskip]
}}

\newtheorem{Definition}{Définition}

\usepackage{siunitx}
\sisetup{
    detect-all,
    output-decimal-marker={,},
    group-minimum-digits = 3,
    group-separator={~},
    number-unit-separator={~},
    inter-unit-product={~}
}

\setlength{\columnseprule}{1pt}


\begin{document}

\textbf{Nom, Prénom :} \hspace{8cm} \textbf{Classe :} \hspace{3cm} \textbf{Date :}\\

\subsection*{DM - Appolo.}

À partir d'une recherche sur Internet. On utilisera principalement Wikipédia :

\url{https://fr.wikipedia.org/wiki/Programme_Apollo#Les_missions_lunaires}


\begin{enumerate}
  \item[1.]Que signifie les lettres N. A. S. et A. dans NASA ? \\
  \Pointilles[2]

  \item[2.]Appolo est le nom du \dotfill
  
  \item[3.]Quelles sont les 7 programmes Appolo qui se sont posés sur la lune ? \\
  \Pointilles[4]

  \item[4.]Qui sont les deux premiers astronautes à avoir marché sur la Lune ? Donner la date.\\
  \Pointilles[1]

  Quelle phrase a rendu célèbre Armstrong lors de ses premiers pas sur la Lune ? \\
  \Pointilles[2]

  Et en anglais. \\
  \Pointilles[2]

  \item[5.]Combien de personnes ont marché sur la Lune ? \\
  \Pointilles[3]
\end{enumerate}

\subsection*{Et la suite ?}

À partir d'une recherche à l'aide d'un moteur de recherche. 

\begin{itemize}
  \item On utilisera les mots clés : \textbf{Prochaine mission lune}
  \item Dès qu'on connaît le nom de la mission, on peut l'utiliser directement dans le moteur de recherche.
\end{itemize}
\url{https://www.asc-csa.gc.ca/fra/astronomie/exploration-lune/missions-artemis.asp}

\begin{enumerate}
  \item[a.] Quel est le nom de la prochaine mission sur la Lune ? \dotfill
  \item[b.] Quelle est la date du premier vol d'essai habité ? \dotfill
  \item[c.] Quelle est la date où les astronautes se poseront sur la Lune ? \dotfill
  \item[d.] Combien de temps le voyage dure-t-il ? \dotfill
\end{enumerate}

\newpage
\textbf{Nom, Prénom :} \hspace{8cm} \textbf{Classe :} \hspace{3cm} \textbf{Date :}\\

\subsection*{DM - Appolo.}

À partir d'une recherche sur Internet. On utilisera principalement Wikipédia :

\url{https://fr.wikipedia.org/wiki/Programme_Apollo#Les_missions_lunaires}


\begin{enumerate}
  \item[1.]Que signifie les lettres N. A. S. et A. dans NASA ? \\
  \Pointilles[2]

  \item[2.]Appolo est le nom du \dotfill
  
  \item[3.]Quelles sont les 7 programmes Appolo qui se sont posés sur la lune ? \\
  \Pointilles[4]

  \item[4.]Qui sont les deux premiers astronautes à avoir marché sur la Lune ? Donner la date.\\
  \Pointilles[1]

  Quelle phrase a rendu célèbre Armstrong lors de ses premiers pas sur la Lune ? \\
  \Pointilles[2]

  Et en anglais. \\
  \Pointilles[2]

  \item[5.]Combien de personnes ont marché sur la Lune ? \\
  \Pointilles[3]
\end{enumerate}

\subsection*{Et la suite ?}

À partir d'une recherche à l'aide d'un moteur de recherche. 

\begin{itemize}
  \item On utilisera les mots clés : \textbf{Prochaine mission lune}
  \item Dès qu'on connaît le nom de la mission, on peut l'utiliser directement dans le moteur de recherche.
\end{itemize}
\url{https://www.asc-csa.gc.ca/fra/astronomie/exploration-lune/missions-artemis.asp}

\begin{enumerate}
  \item[a.] Quel est le nom de la prochaine mission sur la Lune ? \dotfill
  \item[b.] Quelle est la date du premier vol d'essai habité ? \dotfill
  \item[c.] Quelle est la date où les astronautes se poseront sur la Lune ? \dotfill
  \item[d.] Combien de temps le voyage dure-t-il ? \dotfill
\end{enumerate}

\end{document}
