\documentclass[11pt]{article}
\usepackage{geometry,marginnote} % Pour passer au format A4
\geometry{hmargin=1cm, vmargin=1.5cm} % 

% Page et encodage
\usepackage[T1]{fontenc} % Use 8-bit encoding that has 256 glyphs
\usepackage[english,french]{babel} % Français et anglais
\usepackage[utf8]{inputenc} 

\usepackage{lmodern}
\usepackage[np]{numprint}
\setlength\parindent{0pt}

% Graphiques
\usepackage{graphicx,float,grffile}
\usepackage{tikz,pst-eucl,pst-plot,pstricks,pst-node,pstricks-add,pst-fun,pgfplots} 

% Maths et divers
\usepackage{amsmath,amsfonts,amssymb,amsthm,verbatim,scratch3}
\usepackage{multicol,enumitem,url,eurosym,gensymb,tabularx}

\DeclareUnicodeCharacter{20AC}{\euro}



% Sections
\usepackage{sectsty} % Allows customizing section commands
\allsectionsfont{\centering \normalfont\scshape}

% Tête et pied de page
\usepackage{fancyhdr} \pagestyle{fancy} \fancyhead{} \fancyfoot{}

%\fancyfoot[L]{Collège Faubert}
%\fancyfoot[C]{\thepage / 6}
%\fancyfoot[R]{Série Générale}

\renewcommand{\headrulewidth}{0pt} % Remove header underlines
%\renewcommand{\footrulewidth}{0pt} % Remove footer underlines

\newcommand{\horrule}[1]{\rule{\linewidth}{#1}} % Create horizontal rule command with 1 argument of height

\newcommand{\Pointilles}[1][3]{%
  \multido{}{#1}{\makebox[\linewidth]{\dotfill}\\[\parskip]
}}

\newtheorem{Definition}{Définition}

\usepackage{siunitx}
\sisetup{
    detect-all,
    output-decimal-marker={,},
    group-minimum-digits = 3,
    group-separator={~},
    number-unit-separator={~},
    inter-unit-product={~}
}

\setlength{\columnseprule}{1pt}


\begin{document}

\begin{center}
  \textit{Pourquoi apprendre alors que l’ignorance est instantanée ?} - \textbf{Bill Watterson}
\end{center}

\subsection*{Exercice 1} 
On sait que $17 \times 29 = 493$. \\
Réciter les deux phrases de cours : multiples et diviseurs. 


\subsection*{Exercice 2} 

\begin{enumerate}
  \item[1.] Calculer les six premiers multiples de 41.
  \item[2.] Calculer le centième multiple de 123.
  \item[3.] Calculer le plus petit multiple de 37 à être plus grand que 12 600.
  \item[4.] 650 est-il un multiple de 13 ?   
\end{enumerate}

\subsection*{Exercice 3} 

\begin{enumerate}
  \item[1.] 137 est-il un diviseur de 6987 ? 
  \item[2.] 254 est-il un diviseur de 4317 ? 
  \item[3.] Calculer tous les diviseurs de 130.
  \item[4.] Calculer le plus grand diviseur commun entre 210 et 63.
\end{enumerate}

\subsection*{Exercice 4} 

\begin{enumerate}
  \item[1.] Poser la division euclidienne de 5424 par 15. 
  \item[2.] Recopier et Compléter : \newline
  Quotient = $\ldots\ldots\ldots$ Reste = $\ldots\ldots\ldots$ \newline
  5424 = $\ldots\ldots\ldots \times 15 + \ldots\ldots\ldots$
  \item[3.] En déduire : \newline
  5425 = $\ldots\ldots\ldots \times 15 + \ldots\ldots\ldots$
\end{enumerate}

\subsection*{PB1 - Fleuriste} 
Dans sa boutique, un fleuriste se concentre sur la création de bouquets avec 4 roses rouges et 5 roses blanches. 
Il a reçu une commande pour 135 bouquets.

\begin{itemize}[label={$\bullet$}]
  \item Rose rouge : 1,10€. 
  \item Rose blanche : 1,25€. 
\end{itemize}

\begin{enumerate}
  \item[1.] Calculer le prix de vente d'un bouquet. 
  \item[2.] Calculer le nombre de roses rouges et le nombre de roses banches nécessaires pour réaliser cette commande.
  \item[3.] Calculer le prix de la commande de 135 bouquets. 
\end{enumerate}

\subsection*{PB2 - Cadeaux de fin d'année} 
\textsc{M. LAFOND} décide de faire un cadeau en fin d'année à chaque élève de sa classe de 5è favorite composée de 23 élèves. Pour se faire il commande 725 cartes Pokemon et 634 chocobons. Il décide de faire un paquet cadeau avec le plus de cartes Pokemon et le plus de chocobons pour chacun des élèves de la classe et comme il n'a pas de chouchou, chacun va en recevoir autant. 


\begin{enumerate}
  \item[1.] Combien de cartes Pokemon et de chocobons va recevoir chaque élève ? 
  \item[2.] Lorsqu'il a fini de faire les cadeaux aux élèves, combien de cartes Pokemon et de chocobons va-t-il lui rester pour donner à ses enfants ?

\end{enumerate}

\end{document}