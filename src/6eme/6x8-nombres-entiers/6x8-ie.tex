\documentclass[11pt]{article}
\usepackage{geometry,marginnote} % Pour passer au format A4
\geometry{hmargin=1cm, vmargin=1.5cm} % 

% Page et encodage
\usepackage[T1]{fontenc} % Use 8-bit encoding that has 256 glyphs
\usepackage[english,french]{babel} % Français et anglais
\usepackage[utf8]{inputenc} 

\usepackage{lmodern}
\usepackage[np]{numprint}
\setlength\parindent{0pt}

% Graphiques
\usepackage{graphicx,float,grffile}
\usepackage{tikz,pst-eucl,pst-plot,pstricks,pst-node,pstricks-add,pst-fun,pgfplots} 

% Maths et divers
\usepackage{amsmath,amsfonts,amssymb,amsthm,verbatim,scratch3}
\usepackage{multicol,enumitem,url,eurosym,gensymb,tabularx}

\DeclareUnicodeCharacter{20AC}{\euro}



% Sections
\usepackage{sectsty} % Allows customizing section commands
\allsectionsfont{\centering \normalfont\scshape}

% Tête et pied de page
\usepackage{fancyhdr} \pagestyle{fancy} \fancyhead{} \fancyfoot{}

%\fancyfoot[L]{Collège Faubert}
%\fancyfoot[C]{\thepage / 6}
%\fancyfoot[R]{Série Générale}

\renewcommand{\headrulewidth}{0pt} % Remove header underlines
%\renewcommand{\footrulewidth}{0pt} % Remove footer underlines

\newcommand{\horrule}[1]{\rule{\linewidth}{#1}} % Create horizontal rule command with 1 argument of height

\newcommand{\Pointilles}[1][3]{%
  \multido{}{#1}{\makebox[\linewidth]{\dotfill}\\[\parskip]
}}

\newtheorem{Definition}{Définition}

\usepackage{siunitx}
\sisetup{
    detect-all,
    output-decimal-marker={,},
    group-minimum-digits = 3,
    group-separator={~},
    number-unit-separator={~},
    inter-unit-product={~}
}

\setlength{\columnseprule}{1pt}


\begin{document}

\textbf{Nom, Prénom :} \hspace{8cm} \textbf{Classe :} \hspace{3cm} \textbf{Date :}\\

\begin{center}
  \textit{L’imagination est une grande puissance dont généralement on ne tient pas assez compte dans la société.} \\ 
  \textbf{Mikhaïl Bakounine}
\end{center}

\subsection*{Ex1 - Les multiples}

\begin{enumerate}
  \item[1a.] Calculer les 12 premiers multiples de 21.
  \item[1b.] Calculer les 5 premiers multiples de 42 qui sont plus grands que 500.
  \item[1c.] 36 est-il un multiples de 720 ?
\end{enumerate}

\subsection*{Les diviseurs}

\begin{enumerate}
  \item[2a.] 45 est-il un diviseur de 765 ?
  \item[2b.] Calculer tous les diviseurs de 80. Écrire les multiplications. 
  \item[2c.] Quel est le nombre de diviseur de 80 ?
  \item[2d.] Calculer tous les diviseurs de 120. Écrire les multiplications. 
  \item[2e.] Quel est le nombre de diviseur de 120 ?
  \item[2f.] Soient $a$ et $b$ deux nombres entiers. Si $b$ est un multiple de $a$. \\
   $a$ est-il toujours un diviseur de $b$ ?
\end{enumerate}

\subsection*{Les critères de divisibilité : Vrai ou Faux ?}

Conseil de rédaction : Rayer lorsque les nombres qui ne sont pas des diviseurs.

\begin{enumerate}
  \item[3a.] Le nombre 24 est-il divisible par : 2 ; 3 ; 5 ; 6 ; 12 ; 15 ?
  \item[3a.] Le nombre 100 est-il divisible par : 2 ; 3 ; 5 ; 6 ; 12 ; 15 ?
  \item[3a.] Le nombre 36 est-il divisible par : 2 ; 3 ; 5 ; 6 ; 12 ; 15 ?
  \item[3a.] Le nombre 44 est-il divisible par : 2 ; 3 ; 5 ; 6 ; 12 ; 15 ?
  \item[3a.] Le nombre 41 est-il divisible par : 2 ; 3 ; 5 ; 6 ; 12 ; 15 ?
\end{enumerate}

\subsection*{Les nombres premiers}

\begin{enumerate}
  \item[4a.] Donner la définition d'un nombre premier. 
  \item[4b.] 25 est-il premier ?
  \item[4c.] 31 est-il premier ?
  \item[4d.] 391 est-il premier ?
  \item[4e.] 353 est-il premier ? 
\end{enumerate}


\end{document}