\documentclass[11pt]{article}
\usepackage{geometry,marginnote} % Pour passer au format A4
\geometry{hmargin=1cm, vmargin=1.5cm} % 

% Page et encodage
\usepackage[T1]{fontenc} % Use 8-bit encoding that has 256 glyphs
\usepackage[english,french]{babel} % Français et anglais
\usepackage[utf8]{inputenc} 

\usepackage{lmodern}
\usepackage[np]{numprint}
\setlength\parindent{0pt}

% Graphiques
\usepackage{graphicx,float,grffile}
\usepackage{tikz,pst-eucl,pst-plot,pstricks,pst-node,pstricks-add,pst-fun,pgfplots} 

% Maths et divers
\usepackage{amsmath,amsfonts,amssymb,amsthm,verbatim,scratch3}
\usepackage{multicol,enumitem,url,eurosym,gensymb,tabularx}

\DeclareUnicodeCharacter{20AC}{\euro}



% Sections
\usepackage{sectsty} % Allows customizing section commands
\allsectionsfont{\centering \normalfont\scshape}

% Tête et pied de page
\usepackage{fancyhdr} \pagestyle{fancy} \fancyhead{} \fancyfoot{}

%\fancyfoot[L]{Collège Faubert}
%\fancyfoot[C]{\thepage / 6}
%\fancyfoot[R]{Série Générale}

\renewcommand{\headrulewidth}{0pt} % Remove header underlines
%\renewcommand{\footrulewidth}{0pt} % Remove footer underlines

\newcommand{\horrule}[1]{\rule{\linewidth}{#1}} % Create horizontal rule command with 1 argument of height

\newcommand{\Pointilles}[1][3]{%
  \multido{}{#1}{\makebox[\linewidth]{\dotfill}\\[\parskip]
}}

\newtheorem{Definition}{Définition}

\usepackage{siunitx}
\sisetup{
    detect-all,
    output-decimal-marker={,},
    group-minimum-digits = 3,
    group-separator={~},
    number-unit-separator={~},
    inter-unit-product={~}
}

\setlength{\columnseprule}{1pt}


\begin{document}

\textbf{Nom, Prénom :} \hspace{8cm} \textbf{Classe :} \hspace{3cm} \textbf{Date :}\\

\begin{center}
  \textit{La normalité est une route pavée : on y marche aisément mais les fleurs n’y poussent pas.} \\ 
  \textbf{Vincent Van Gogh}
\end{center}

\subsubsection*{Définition} 

Le périmètre d'une figure est \dotfill \\ \Pointilles[1]

\subsubsection*{Ex1 - Calculer et donner la valeur approchée.} 

\begin{multicols}{2}\noindent
\begin{itemize}[label={$\bullet$}]
  \item $\dfrac{2}{3} + 10$ \dotfill \\
  \item $25 \times \pi$ \dotfill \\
  \item $2 \times 51 \div 37$ \dotfill \\
  \item $100 + \pi$ \dotfill \\
\end{itemize} \end{multicols}

\subsubsection*{Ex2 - Arrondir aux centièmes} 

\begin{multicols}{2}\noindent
\begin{itemize}[label={$\bullet$}]
  \item $35,348$ \dotfill \\
  \item $12,125$ \dotfill \\
  \item $27,402$ \dotfill \\
  \item $21,997$ \dotfill \\
\end{itemize} \end{multicols}

\subsubsection*{Ex3 - Calculer les périmètres} 

\begin{multicols}{2}\noindent
\begin{figure}[H]
  \centering
  \includegraphics[width=0.8\linewidth]{6x9-perimetre/exo3a.pdf}
\end{figure} \columnbreak

\begin{figure}[H]
  \centering
  \includegraphics[width=\linewidth]{6x9-perimetre/exo3b.pdf}
\end{figure} 

\end{multicols}

\Pointilles[15] 

\newpage

\subsubsection*{Ex4 - Calculer les périmètres} 

\begin{multicols}{2}\noindent

\begin{figure}[H]
  \centering
  \includegraphics[width=.9\linewidth]{6x9-perimetre/exo3c.pdf}
\end{figure} \columnbreak

\Pointilles[10] 

\end{multicols}




\subsubsection*{Pb1 : Bouquets} 

Un bouquet de fleurs est composé de trois Belladone, six Aconit, deux Iris et une Narcisse. Je souhaite acheter 32 bouquets.

Au total, combien ai-je de fleur ?

\Pointilles[6] 


\subsubsection*{Pb2 : Carte Pokémon} 

Je possède une belle collection de 281 cartes Pokémon. j’achète un classeur avec des pochettes pouvant contenir 16 cartes.

Quelle quantité de pochette est nécessaire pour ranger toute mes cartes ?

\Pointilles[6]


\subsubsection*{Pb3 : Bamboche} 

Pour ma soirée pyjama, j'achète trois bouteilles de jus de pommes à 1,40€, un paquet de chips à 4,90€ et deux paquets de bonbons à 6,60€. Je paye en donnant un billet de 20€ et un billet de 10€. 

Quelle somme le caissier doit-il me rendre ?

\Pointilles[6]



\end{document}